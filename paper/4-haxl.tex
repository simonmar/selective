\section{Speculative execution}\label{sec-haxl}

\Haxl\cite{haxl} is a framework for efficiently executing code that
fetches data from external sources, typically databases or remote
services. The \Haxl framework allows such code to be written in a
natural style using \hs{Applicative} and \hs{Monad} combinators, while
automatically parallelising the data fetch operations and batching
together multiple fetches from the same data source.

To illustrate the idea using a fragment of the example code from
\citep{Haxl}, suppose we're writing the code to render a blog into
HTML. The blog consists of a set of posts, where each post is
identified by a \hs{PostId}.  The data for the blog is stored in a
remote database.  The API for fetching the data from the database is
as follows:

\begin{minted}[fontsize=\small]{haskell}
getPostIds     :: Haxl [PostId]
getPostContent :: PostId -> Haxl PostContent
\end{minted}

We can fetch the set of all \hs{PostId}s using \hs{getPostIds}, and we
can fetch the content of one post using \hs{getPostContent}.

To get the content of all posts we could write:

\begin{minted}[fontsize=\small]{haskell}
getAllPostsContent :: Haxl [PostContent]
getAllPostsContent = mapM getPostContent =<< getPostIds
\end{minted}

Now, when we \hs{apM getPostContent} we would really like those
database queries to happen in parallel, since there are no
dependencies between them. Furthermore, we might even be able to batch
up the queries into a single request to the remote database.

These optimisations are performed automatically by \Haxl, using a
special instance of \hs{Applicative} that exploits the lack of dependency
between the two computations to explore the computations and collect
the data fetch operations that can be performed in parallel or batched
together.

\Haxl has been in use at Facebook, at scale, to proactively detect and
remediate various forms of abuse for several years now.  One of the
key tools that we found to be useful in the kind of code we write is
the ``lazy'' conditional operators:

\begin{minted}[fontsize=\small]{haskell}
(.||) :: Haxl Bool -> Haxl Bool -> Haxl Bool
x .|| y = do b <- x; if b then return True else y

(.&&) :: Haxl Bool -> Haxl Bool -> Haxl Bool
x .&& y = do b <- x; if b then y else return False
\end{minted}

These are typically used to improve performance by guarding expensive
checks with cheaper checks.  For example, we might write:

\begin{minted}[fontsize=\small]{haskell}
myDetector =
  if cheapCondition .&& expensiveCondition
    then ...
    else ...
\end{minted}

\noindent The idea is that \hs{cheapCondition} is cheap to evaluate and
returns \hs{False} in a large proportion of cases, so that we can
often avoid needing to perform \hs{expensiveCondition}.

This is all fine and doesn't require any additional extensions or
special support in Haxl. But we also noticed that sometimes there is a
pair of conditions where neither is obviously cheaper than the other,
yet we would still like to benefit from bailing out early when the
answer is known.  Therefore \Haxl contains two more conditional
operators \hs{pOr} and \hs{pAnd} for ``parallel Or'' and ``parallel
And'' respectively:

\begin{minted}[fontsize=\small]{haskell}
pOr, pAnd :: Haxl Bool -> Haxl Bool -> Haxl Bool
\end{minted}

These have the behaviour that

\begin{itemize}
\item Both arguments are evaluated in parallel.
\item The computation is aborted as soon as the answer is known, even
  if the other argument has not completed evaluation yet.
\end{itemize}

Data-fetches are not observable effects, so the parallelism is not
observable to the programmer (this is the property that \Haxl relies
on for the soundness of its parallel \hs{Applicative}
instance). However, \hs{pOr} and \hs{pAnd} are non-deterministic with
respect to exceptions: if an exception is thrown by either side, then
it will be thrown by the computation as a whole immediately without
waiting for the other side to complete.  One could imagine an
alternative implementation which waits for the completion of the other
argument when an exception is raised; this would be deterministic, but
would be less efficient in the case of exceptions.

We found that \hs{pOr} and \hs{pAnd} are not only useful by
themselves, but also when chaining together a set of conditions:

\begin{minted}[fontsize=\small]{haskell}
pAny :: (a -> Haxl Bool) -> [a] -> Haxl Bool
pAny f (x:xs) = f x `pOr` pAny f xs
pAny _ [] = return False
\end{minted}

Indeed when there are a large number of conditions it becomes very
hard to manually order them effectively using \hs{.||}, especially
when the set of conditions changes often. In this case \hs{pAny} is
very useful as a way to extract performance in a non-invasive way.

It should come as no surprise that \hs{pOr} and \hs{pAnd} can be
implemented using \hs{select}, indeed \hs{pOr = (<||>)} and
\hs{pAnd = (<&&>)}.

An implementation of \hs{select} in terms of the \Haxl implementation
from \citet{haxl} is given in Figure~\ref{fig-haxl-select}.  For the
purposes of the presentation here we have renamed \hs{Fetch} to
\hs{Haxl} and added support for exceptions in the form of the
\hs{Throw} cases.

\begin{figure}
\begin{minted}[fontsize=\small]{haskell}
instance Selective Haxl where
  select (Haxl x) (Haxl f) = Haxl $ do
    rx <- x
    case rx of
      Done (Right b) -> return (Done b)
      Done (Left  a) -> unHaxl (($a) <$> Haxl f)
      Throw e -> return (Throw e)
      Blocked br1 c -> do
        rf <- f
        case rf of
          Done f  -> unHaxl (either f id <$> c)
          Throw e -> return (Throw e)
          Blocked br2 d -> return (Blocked (br1 <> br2) (select c d)
\end{minted}
\label{fig-haxl-select}
\end{figure}

There is one wrinkle in the implementation of \hs{pOr} and \hs{pAnd}
in terms of \hs{select}. Ideally, \hs{pOr} and \hs{pAnd} would be
symmetric: just as we can abort the right argument if the left
argument determines the answer, we should be able to abort the left
argument in the same way. Yet \hs{select} is inherently left-biased,
so our implementations of \hs{pOr} and \hs{pAnd} using \hs{select} are
suboptimal.  (TODO: what to do about this?)
